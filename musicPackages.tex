\usepackage[brazil]{babel}
\usepackage[sfdefault]{GoSans}
\usepackage[T1]{fontenc}

\usepackage[hidelinks]{hyperref}

\usepackage{ifthen}


% Pesquisar sobre leadsheets latex
\usepackage[full]{leadsheets}
\setleadsheets{chord-cs =\bfseries}
% configurar página
\usepackage[bottom=1cm,top=1cm,left=1cm,right=1cm]{geometry}

% configurar tamanho das letras
% \renewcommand{\normalsize}{\fontsize{11}{16}\selectfont}

% configurar seções
\usepackage{xcolor}
\usepackage{titlesec}    
    \setcounter{secnumdepth}{0}
    % \setcounter{subsecnumdepth}{0}
    \titleformat{\section}{\huge\bfseries}{\thesection}{1em}{}
    \titlespacing\section{0pt}{12pt plus 4pt minus 2pt}{0pt plus 2pt minus 2pt}
    \titleformat{\subsection}{\small\color{gray}}{\thesubsection}{1em}{}

% Box modelo %%%%%%%%%%%%%%%%%%%%%%%%%%%%%%%
\usepackage{multicol}
\usepackage[xparse, skins, breakable]{tcolorbox}
\usepackage{tikz}
\usetikzlibrary{shapes.geometric,arrows,positioning,fit,calc,}
\usepackage{pgf}
\usepackage{varwidth}

\def \ifempty#1{\def\temp{#1} \ifx\temp\empty }

\def \mymacro#1{\ifempty{#1} \else \node[circle,
fill=white,
inner sep=0pt,
draw=black,
minimum size=.45cm,
ultra thick,
align=center,
text=black,
font=\sffamily\bfseries
] at (0.35,0.35) {\tiny#1}; \fi }

\newcommand{\sequence}[2]{
  \begin{tikzpicture}
    \node[circle,
    inner sep=0pt,
    minimum size=.9cm,
    fill=black,
    draw=black,
    baseline = (current bounding box.center), %anchor = north,
    ultra thick,
    align=center,
    text=white,
    text width=.7cm,
    % label=right:{\color{black}{\large #1}},
    font=\sffamily\bfseries\small
    ] {\ifthenelse{\equal{#1}{V1}}
        {V}
        {\ifthenelse{\equal{#1}{R1}}
            {R}
            {\ifthenelse{\equal{#1}{It1}}
                {It}
                {#1}
                }}};
    \mymacro{#2}
  \end{tikzpicture}
  }

\newtcolorbox{descriptionBox}[2][]{
    colback=white,
    colframe=black!35!white,
    fonttitle=\bfseries,
    enhanced jigsaw,
    vfill before first, 
    pad at break=1mm,
    breakable,
    opacityback=1,
    opacitybacktitle=0,
    center title,
    varwidth boxed title*=-4cm,
    % tikznode boxed title,
    % minipage boxed title*=-7cm,
    boxed title style = {%width=3cm,
    colback=white,
    halign=center,valign=center,
    % colframe=white!100!black,
     boxrule=0mm,
     colframe=white,
    %  borderline={0.5mm}{0mm}{white!100!black},
    },
    attach boxed title to top left={yshift=-5.mm, xshift=2mm, yshifttext=-4mm},
    title={
      
      \begin{tikzpicture}
        \node[circle,
         fill=black,
         draw=black,
         inner sep=0pt,
         minimum size=.6cm,
         baseline = (current bounding box.center), %anchor = north,
         ultra thick,
         align=center,
         text=white,
         text width=.45cm,
        %  text height = .45cm,
         label=right:{\color{black}{#1}},
         font=\sffamily\bfseries
         ] (0,0) {\scriptsize#2};
    \end{tikzpicture}} }


    \newcommand{\makepage}{
      \pagestyle{empty}
      \section{\href{\link}{\music}}
      \ifempty{\autor} 
      \else 
        \subsection{
          \autor 
        } 
      \fi
      
      
      \hfill
      \begin{minipage}{5cm}
          \begin{center}
            {\small
              \color{gray}
              Tom: \tom \\
              Andamento: \bpm\, \compasso}
          \end{center}
      \end{minipage}
      
      \vspace{.3cm}
    }

    \newcommand{\makedescription}{
        \begin{multicols}{2}
            \ifthenelse{\equal{\introducao}{}}{}{
            \begin{descriptionBox}[Introdução]{I}
                \introducao
            \end{descriptionBox}}
            
            \ifthenelse{\equal{\versoOne}{}}{}{
            \begin{descriptionBox}[Verso]{V}
              \versoOne
            \end{descriptionBox}}

            \ifthenelse{\equal{\versoTwo}{}}{}{
            \begin{descriptionBox}[Verso 2]{V2}
              \versoTwo
            \end{descriptionBox}}

            \ifthenelse{\equal{\versoThree}{}}{}{
            \begin{descriptionBox}[Verso 3]{V3}
              \versoThree
            \end{descriptionBox}}
        
            \ifthenelse{\equal{\refraoOne}{}}{}{
            \begin{descriptionBox}[Refrão]{R}
              \refraoOne
            \end{descriptionBox}}

            \ifthenelse{\equal{\refraoTwo}{}}{}{
            \begin{descriptionBox}[Refrão 2]{R2}
              \refraoTwo
            \end{descriptionBox}}
        
            \ifthenelse{\equal{\refraoThree}{}}{}{
            \begin{descriptionBox}[Refrão 3]{R3}
              \refraoThree
            \end{descriptionBox}}

            \ifthenelse{\equal{\ponte}{}}{}{
            \begin{descriptionBox}[Ponte]{P}
              \ponte
            \end{descriptionBox}}

            \ifthenelse{\equal{\instrumentalOne}{}}{}{
            \begin{descriptionBox}[Instrumental]{It}
              \instrumentalOne
            \end{descriptionBox}}
            
            \ifthenelse{\equal{\instrumentalTwo}{}}{}{
            \begin{descriptionBox}[Instrumental 2]{It2}
              \instrumentalTwo
            \end{descriptionBox}}

            \ifthenelse{\equal{\instrumentalThree}{}}{}{
            \begin{descriptionBox}[Instrumental 3]{It3}
              \instrumentalThree
            \end{descriptionBox}}

            \ifthenelse{\equal{\fim}{}}{}{
            \begin{descriptionBox}[Fim]{F}
              \fim
            \end{descriptionBox}}
        \end{multicols}

    }