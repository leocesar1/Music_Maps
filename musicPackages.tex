\usepackage[brazil]{babel}
% \usepackage[sfdefault]{GoSans}
\usepackage[sfdefault,condensed]{cabin}
\usepackage{enumitem}
% \usepackage[T1]{fontenc}

% \usepackage[hidelinks]{hyperref}

\usepackage{ifthen}


% Pesquisar sobre leadsheets latex
\usepackage{leadsheets}
\setleadsheets{
  title-template = minlouvor
}

\definesongtitletemplate{minlouvor}{
  \pagestyle{empty}
  \section{\songproperty{title}}
  \ifsongproperty{interpret}
      {\subsection{\songproperty{interpret}}}
      {}

  \hfill
  \begin{minipage}{5cm}
      \begin{center}
          {\small
          \color{gray}
          Tom: \songproperty{key} \\
          \ifsongproperty{tempo}
          {Andamento: \songproperty{tempo}\, \compasso}
          {}%
          }
      \end{center}
  \end{minipage}
  
  \vspace{.3cm}
  
  \makesequence
  }


\defineversetypetemplate{introducao}
  {\descriptionBoxTwo[\versename]{I}
          \ifobeylines
              {\setleadsheets{obey-lines-parskip=\parskip }\setlength\parskip{0pt}}
              {\vspace*{-\parskip}}%
  }
  {\enddescriptionBoxTwo}

\defineversetypetemplate{verso}
    {\descriptionBox[\versename]{V}
            \ifobeylines
                {\setleadsheets{obey-lines-parskip=\parskip }\setlength\parskip{0pt}}
                {\vspace*{-\parskip}}%
    }
    {\enddescriptionBox}  

\defineversetypetemplate{versoDois}
  {\descriptionBox[\versename]{V2}
          \ifobeylines
              {\setleadsheets{obey-lines-parskip=\parskip }\setlength\parskip{0pt}}
              {\vspace*{-\parskip}}%
  }
  {\enddescriptionBox}  

\defineversetypetemplate{versoTres}
  {\descriptionBox[\versename]{V3}
          \ifobeylines
              {\setleadsheets{obey-lines-parskip=\parskip }\setlength\parskip{0pt}}
              {\vspace*{-\parskip}}%
  }
  {\enddescriptionBox}

\defineversetypetemplate{preRefrao}
  {\descriptionBox[\versename]{Pr}
          \ifobeylines
              {\setleadsheets{obey-lines-parskip=\parskip }\setlength\parskip{0pt}}
              {\vspace*{-\parskip}}%
  }
  {\enddescriptionBox}

\defineversetypetemplate{refrao}
  {\descriptionBox[\versename]{R}
          \ifobeylines
              {\setleadsheets{obey-lines-parskip=\parskip }\setlength\parskip{0pt}}
              {\vspace*{-\parskip}}%
  }
  {\enddescriptionBox}

\defineversetypetemplate{refraoDois}
  {\descriptionBox[\versename]{R2}
          \ifobeylines
              {\setleadsheets{obey-lines-parskip=\parskip }\setlength\parskip{0pt}}
              {\vspace*{-\parskip}}%
  }
  {\enddescriptionBox}

\defineversetypetemplate{refraoTres}
  {\descriptionBox[\versename]{R3}
          \ifobeylines
              {\setleadsheets{obey-lines-parskip=\parskip }\setlength\parskip{0pt}}
              {\vspace*{-\parskip}}%
  }
  {\enddescriptionBox}

\defineversetypetemplate{preRefrao}
  {\descriptionBoxTwo[\versename]{Pr}
          \ifobeylines
              {\setleadsheets{obey-lines-parskip=\parskip }\setlength\parskip{0pt}}
              {\vspace*{-\parskip}}%
  }
  {\enddescriptionBoxTwo}

\defineversetypetemplate{preRefraoDois}
  {\descriptionBoxTwo[\versename]{Pr2}
          \ifobeylines
              {\setleadsheets{obey-lines-parskip=\parskip }\setlength\parskip{0pt}}
              {\vspace*{-\parskip}}%
  }
  {\enddescriptionBoxTwo}

\defineversetypetemplate{ponte}
  {\descriptionBox[\versename]{P}
          \ifobeylines
              {\setleadsheets{obey-lines-parskip=\parskip }\setlength\parskip{0pt}}
              {\vspace*{-\parskip}}%
  }
  {\enddescriptionBox}

\defineversetypetemplate{interludio}
  {\descriptionBoxTwo[\versename]{It}
          \ifobeylines
              {\setleadsheets{obey-lines-parskip=\parskip }\setlength\parskip{0pt}}
              {\vspace*{-\parskip}}%
  }
  {\enddescriptionBoxTwo}

\defineversetypetemplate{instrumental}
  {\descriptionBoxTwo[\versename]{In}
          \ifobeylines
              {\setleadsheets{obey-lines-parskip=\parskip }\setlength\parskip{0pt}}
              {\vspace*{-\parskip}}%
  }
  {\enddescriptionBoxTwo}

\defineversetypetemplate{instrumentalDois}
  {\descriptionBoxTwo[\versename]{In2}
          \ifobeylines
              {\setleadsheets{obey-lines-parskip=\parskip }\setlength\parskip{0pt}}
              {\vspace*{-\parskip}}%
  }
  {\enddescriptionBoxTwo}

\defineversetypetemplate{instrumentalTres}
  {\descriptionBoxTwo[\versename]{In3}
          \ifobeylines
              {\setleadsheets{obey-lines-parskip=\parskip }\setlength\parskip{0pt}}
              {\vspace*{-\parskip}}%
  }
  {\enddescriptionBoxTwo}

\defineversetypetemplate{fim}
  {\descriptionBox[\versename]{F}
          \ifobeylines
              {\setleadsheets{obey-lines-parskip=\parskip }\setlength\parskip{0pt}}
              {\vspace*{-\parskip}}%
  }
  {\enddescriptionBox}

\newversetype{introducao}[
  name={Introdução},
  after-label=\noindent,
  template=introducao
  ]
\newversetype{verso}[
  name={Verso},
  after-label=\noindent,
  template=verso
  ]
\newversetype{versoDois}[
  name={Verso 2},
  after-label=\noindent,
  template=versoDois
  ]
\newversetype{versoTres}[
  name={Verso 3},
  after-label=\noindent,
  template=versoTres
  ]

\newversetype{preRefrao}[
  name={Pré Refrão},
  after-label=\noindent,
  template=preRefrao
  ]

\newversetype{preRefraoDois}[
  name={Pré Refrão 2},
  after-label=\noindent,
  template=preRefraoDois
  ]

\newversetype{refrao}[
  name=Refrão,
  after-label=\noindent,
  template=refrao
  ]

\newversetype{refraoDois}[
  name={Refrão 2},
  after-label=\noindent,
  template=refraoDois
  ]
   
\newversetype{refraoTres}[
  name={Refrão 3},
  after-label=\noindent,
  template=refraoTres
  ]

\newversetype{ponte}[
  name=Ponte,
  after-label=\noindent,
  template=ponte
  ]

\newversetype{interludio}[
  name=Interlúdio,
  after-label=\noindent,
  template=interludio
  ]

\newversetype{fim}[
  name=Fim,
  after-label=\noindent,
  template=fim
  ]

\newversetype{instrumental}[
  name=Instrumental,
  after-label=\noindent,
  template=instrumental
  ]

\newversetype{instrumentalDois}[
  name={Instrumental 2},
  after-label=\noindent,
  template=instrumentalDois
  ]
    
\newversetype{instrumentalTres}[
  name={Instrumental 3},
  after-label=\noindent,
  template=instrumentalTres
  ]

% configurar página
\usepackage[bottom=1cm,top=1cm,left=1cm,right=1cm]{geometry}

% configurar tamanho das letras
% \renewcommand{\normalsize}{\fontsize{11}{16}\selectfont}

% configurar seções
\usepackage{xcolor}
\usepackage{titlesec}    
    \setcounter{secnumdepth}{0}
    % \setcounter{subsecnumdepth}{0}
    \titleformat{\section}{\huge\bfseries}{\thesection}{1em}{}
    \titlespacing\section{0pt}{12pt plus 4pt minus 2pt}{0pt plus 2pt minus 2pt}
    \titleformat{\subsection}{\small\color{gray}}{\thesubsection}{1em}{}

% Box modelo %%%%%%%%%%%%%%%%%%%%%%%%%%%%%%%
\usepackage{multicol}
\usepackage[xparse, skins, breakable]{tcolorbox}
\usepackage{tikz}
\usetikzlibrary{shapes.geometric,arrows,positioning,fit,calc,}
\usepackage{pgf}
\usepackage{varwidth}

\newcommand\ifempty[1]{\def\temp{#1} \ifx\temp\empty }

\newcommand\mymacro[1]{\ifempty{#1} \else \node[circle,
fill=white,
inner sep=0pt,
draw=black,
minimum size=.45cm,
ultra thick,
align=center,
text=black,
font=\sffamily\bfseries
] at (0.35,0.35) {\tiny#1}; \fi }

\newcommand{\sequence}[2]{
  \begin{tikzpicture}
    \node[circle,
    inner sep=0pt,
    minimum size=.9cm,
    fill=black,
    draw=black,
    baseline = (current bounding box.center), %anchor = north,
    ultra thick,
    align=center,
    text=white,
    text width=.7cm,
    % label=right:{\color{black}{\large #1}},
    font=\sffamily\bfseries\small
    ] {\ifthenelse{\equal{#1}{V1}}
        {V}
        {\ifthenelse{\equal{#1}{R1}}
            {R}
            {\ifthenelse{\equal{#1}{It1}}
                {It}
                {#1}
                }}};
    \mymacro{#2}
  \end{tikzpicture}
  }

  \newtcolorbox{descriptionBox}[2][]{
    colback=white,
    colframe=black!35!white,
    fonttitle=\bfseries,
    enhanced jigsaw,
    vfill before first, 
    pad at break=1mm,
    breakable,
    opacityback=1,
    opacitybacktitle=0,
    center title,
    varwidth boxed title*=-4cm,
    boxed title style = {
    colback=white,
    halign=center,valign=center,
    boxrule=0mm,
    colframe=white,
    },
    attach boxed title to top left={yshift=-5.mm, xshift=2mm, yshifttext=-4mm},
    title={      
        \begin{tikzpicture}
            \node[circle,
            fill=black,
            draw=black,
            inner sep=0pt,
            minimum size=.6cm,
            baseline = (current bounding box.center),
            ultra thick,
            align=center,
            text=white,
            text width=.45cm,
            label={[xshift=1.0cm, yshift=-0.6cm]:{\color{black}{#1}}},
            font=\sffamily\bfseries
            ] (0,0) {\scriptsize #2};
        \end{tikzpicture}
        }
    }

    \newtcolorbox{descriptionBoxTwo}[2][]{
      colback=white,
      colframe=black!35!white,
      fonttitle=\bfseries,
      enhanced jigsaw,
      vfill before first, 
      pad at break=1mm,
      breakable,
      opacityback=1,
      opacitybacktitle=0,
      center title,
      varwidth boxed title*=-4cm,
      boxed title style = {
      colback=white,
      halign=center,valign=center,
      boxrule=0mm,
      colframe=white,
      },
      attach boxed title to top left={yshift=-5.mm, xshift=2mm, yshifttext=-4mm},
      title={      
          \begin{tikzpicture}
              \node[circle,
              fill=black,
              draw=black,
              inner sep=0pt,
              minimum size=.6cm,
              baseline = (current bounding box.center),
              ultra thick,
              align=center,
              text=white,
              text width=.45cm,
              label={[xshift=1.5cm, yshift=-0.65cm]:{\color{black}{#1}}},
              font=\sffamily\bfseries
              ] (0,0) {\scriptsize #2};
          \end{tikzpicture}
          }
      }

      \newenvironment{anotation}[1]
      {\color{gray}\itemize[
         nosep,
         leftmargin=\dimexpr\textwidth-#1\relax,
         rightmargin=0pt,
         itemindent=\parindent,
         listparindent=\parindent,
       ]\item[]\relax}
      {\enditemize}
      
      \newcommand{\note}[1]{
         \begin{anotation}{5cm}
             #1
         \end{anotation}
      }